%\usepackage[letterpaper]{geometry}
\documentclass[10pt]{article}

\usepackage{hyperref}
\usepackage{graphicx}
\usepackage{balance}
\usepackage{rotating}

\usepackage{tikz}
\usepackage{pgfplots}
\usepackage{pgfplotstable}

\usepackage{times}
\usepackage{amsmath}
\usepackage{amssymb}
\usepackage{float}
\usepackage{subfigure}

\topmargin=-1in % Moves the top of the document 1 inch above the default
\textheight=8.6in % Total height of the text on the page before text goes on to the next page, this can be increased in a longer letter
\oddsidemargin=-3pt % Position of the left margin, can be negative or positive if you want more or less room
\textwidth=6.6in % Total width of the text, increase this if the left margin was decreased and vice-versa

\newcommand{\minisec}[1]{\vspace*{0.05cm}\noindent\textbf{#1.\\}}
\newcommand{\rquote}[2]{%
\begin{description}%
\item[#1]: ``#2''
\end{description}}


\begin{document}
\setlength{\pdfpagewidth}{8.5in}
\setlength{\pdfpageheight}{11in}

\begin{center}
\large\bf Reproducible Results for \\
``Fast Iterative Graph Computation with Block Updates''
\end{center}

\begin{figure*}[ht!]
\begin{tabular}{c|ccc}
& Effect of block size & Effect of block strategy & Effect of InSchedule \\
& (Time vs block size) & (Time vs scheduling policy) & (Time vs
  max inner iterations) \\
\\ \hline
%
% =====================================================================
% PPR(Google) 
%
\begin{sideways} 
  \quad\quad 
  \parbox{18mm}{
    PPR(Google) \\
    {\scriptsize (Default BS=100)}
  }
\end{sideways} 
&
\pgfplotstableread{result/personalpr/blocksize.dat}{\tbltimeaa}
\begin{tikzpicture}
  \begin{axis}[
      ylabel={Run time (s)}, ylabel near ticks,
      ymin={0},
      width={0.3\textwidth},
      height={36mm},
      legend entries={Static,Eager,Prior},
      legend columns=-1,
      legend to name=namedeffect,
    ]
    \addplot table[x index=0,y index=1]{\tbltimeaa};
    \addplot table[x index=0,y index=2]{\tbltimeaa};
    \addplot table[x index=0,y index=3]{\tbltimeaa};
    \draw [gray, very thin] 
    ({axis cs:100,0}|-{rel axis cs:0,0}) -- 
    ({axis cs:100,0}|-{rel axis cs:0,1});
  \end{axis}
\end{tikzpicture} 
&
\pgfplotstableread{result/personalpr/strategy.dat}{\tbltimeab}
\begin{tikzpicture}
  \begin{axis}[
      ybar, ymin=0, bar width=6pt,
      width=0.33\textwidth,
      height=36mm,
      xtick=data,
      xtick align=inside,
      xticklabels from table={\tbltimeab}{Schedule},
      enlarge x limits={0.3},
      legend entries={Vertex,VertexCA,BlockS,BlockCvg},
      legend columns=2,
      legend to name=namedb,
    ]
    \addplot table [x expr=\coordindex+1,y=Vertex  ] {\tbltimeab};
    \addplot table [x expr=\coordindex+1,y=VertexCA] {\tbltimeab};
    \addplot table [x expr=\coordindex+1,y=BlockS  ] {\tbltimeab};
    \addplot table [x expr=\coordindex+1,y=BlockCvg] {\tbltimeab};
  \end{axis}
\end{tikzpicture} 
&
\pgfplotstableread{result/personalpr/maxinner.dat}{\tbltimeac}
\begin{tikzpicture}
  \begin{axis}[
      ymin={0},
      xtick=data,
      width={0.3\textwidth},
      height={36mm},
      %legend entries={Static,StaticCvg,Eager,EagerCvg,Prior,PriorCvg},
      legend entries={Static,Eager,Prior},
      legend columns=3,
      legend to name=namedc,
    ]
    \addplot table[x index=0,y=Static   ]{\tbltimeac};
%    \addplot[blue] table[x index=0,y=StaticCvg]{\tbltimeac};
    \addplot table[x index=0,y=Eager    ]{\tbltimeac};
%    \addplot[red]  table[x index=0,y=EagerCvg ]{\tbltimeac};
    \addplot table[x index=0,y=Prior    ]{\tbltimeac};
%    \addplot[brown]  table[x index=0,y=PriorCvg ]{\tbltimeac};
\end{axis}
\end{tikzpicture} 
\\
%
% =====================================================================
% PPR(DBLP) 
%

%\begin{sideways} 
%  \quad\quad 
%  \parbox{18mm}{
%    PPR(DBLP) \\
%    {\scriptsize (Default BS=100)}
%  }
%\end{sideways} 
%&
%\pgfplotstableread{figs/jdb32G/data/pagerank/blocksize_dblp.dat}{\tbltimeba}
%\begin{tikzpicture}
%  \begin{axis}[
%      ylabel={Run time (s)}, ylabel near ticks,
%      ymin={0},
%      width={0.3\textwidth},
%      height={36mm},
%    ]
%    \addplot table[x index=0,y index=1]{\tbltimeba};
%    \addplot table[x index=0,y index=2]{\tbltimeba};
%    \addplot table[x index=0,y index=3]{\tbltimeba};
%    \draw [gray, very thin]
%    ({axis cs:100,0}|-{rel axis cs:0,0}) -- 
%    ({axis cs:100,0}|-{rel axis cs:0,1});
%  \end{axis}
%\end{tikzpicture} 
%&
%\pgfplotstableread{figs/jdb32G/data/pagerank/breakdown_dblp.dat}{\tbltimebb}
%\begin{tikzpicture}
%  \begin{axis}[
%      ybar, ymin=0, bar width=6pt,
%      width=0.33\textwidth,
%      height=36mm,
%      xtick=data,
%      xtick align=inside,
%      xticklabels from table={\tbltimebb}{Schedule},
%      enlarge x limits={0.3}
%    ]
%    \addplot table [x expr=\coordindex+1,y=Vertex  ] {\tbltimebb};
%    \addplot table [x expr=\coordindex+1,y=VertexCA] {\tbltimebb};
%    \addplot table [x expr=\coordindex+1,y=BlockS  ] {\tbltimebb};
%    \addplot table [x expr=\coordindex+1,y=BlockCvg] {\tbltimebb};
%  \end{axis}
%\end{tikzpicture} 
%&
%\pgfplotstableread{figs/jdb32G/data/pagerank/inner_dblp.dat}{\tbltimebc}
%\begin{tikzpicture}
%  \begin{axis}[
%      ymin={0},
%      xtick=data,
%      width={0.3\textwidth},
%      height={36mm}
%    ]
%    \addplot table[x index=0,y=Static   ]{\tbltimebc};
%%    \addplot table[x index=0,y=StaticCvg]{\tbltimebc};
%    \addplot table[x index=0,y=Eager    ]{\tbltimebc};
%%    \addplot table[x index=0,y=EagerCvg ]{\tbltimebc};
%    \addplot table[x index=0,y=Prior    ]{\tbltimebc};
%%    \addplot table[x index=0,y=PriorCvg ]{\tbltimebc};
%  \end{axis}
%\end{tikzpicture} 


\\
%
% =====================================================================
% SSSP 
%
\begin{sideways} 
  \quad\quad 
  \parbox{18mm}{
    SSSP  \\
    {\scriptsize (Default BS=100)}
  }
\end{sideways} 
&
\pgfplotstableread{result/sssp/blocksize.dat}{\tbltimeca}
\begin{tikzpicture}
  \begin{axis}[
      ylabel={Run time (s)}, ylabel near ticks,
      ymin={0},
      width={0.3\textwidth},
      height={36mm},
    ]
    \addplot table[x index=0,y index=1]{\tbltimeca};
    \addplot table[x index=0,y index=2]{\tbltimeca};
    \addplot table[x index=0,y index=3]{\tbltimeca};
    \draw [gray, very thin]
    ({axis cs:100,0}|-{rel axis cs:0,0}) -- 
    ({axis cs:100,0}|-{rel axis cs:0,1});
  \end{axis}
\end{tikzpicture} 
&
\pgfplotstableread{result/sssp/strategy.dat}{\tbltimecb}
\begin{tikzpicture}
  \begin{axis}[
      ybar, ymin=0, bar width=6pt,
      width=0.33\textwidth,
      height=36mm,
      xtick=data,
      xtick align=inside,
      xticklabels from table={\tbltimecb}{Schedule},
      enlarge x limits={0.3}
    ]
    \addplot table [x expr=\coordindex+1,y=Vertex  ] {\tbltimecb};
    \addplot table [x expr=\coordindex+1,y=VertexCA] {\tbltimecb};
    \addplot table [x expr=\coordindex+1,y=BlockS  ] {\tbltimecb};
    \addplot table [x expr=\coordindex+1,y=BlockCvg] {\tbltimecb};
  \end{axis}
\end{tikzpicture} 
&
\pgfplotstableread{result/sssp/maxinner.dat}{\tbltimecc}
\begin{tikzpicture}
  \begin{axis}[
      ymin={0},
      xtick=data,
      width={0.3\textwidth},
      height={36mm}
    ]
    \addplot table[x index=0,y=Static   ]{\tbltimecc};
%    \addplot table[x index=0,y=StaticCvg]{\tbltimecc};
    \addplot table[x index=0,y=Eager    ]{\tbltimecc};
%    \addplot table[x index=0,y=EagerCvg ]{\tbltimecc};
    \addplot table[x index=0,y=Prior    ]{\tbltimecc};
%    \addplot table[x index=0,y=PriorCvg ]{\tbltimecc};
  \end{axis}
\end{tikzpicture}
\\
%
% =====================================================================
% Etch Sim
%
\begin{sideways} 
  \quad\quad 
  \parbox{18mm}{
    Etch Sim \\
    {\scriptsize (Default BS=125)}
  }
\end{sideways} 
&
\pgfplotstableread{result/eik3d/blocksize.dat}{\tbltimeda}
\begin{tikzpicture}
  \begin{axis}[
      ylabel={Run time (s)}, ylabel near ticks,
      ymin={0},
      width={0.3\textwidth},
      height={36mm},
    ]
    \addplot table[x index=0,y index=1]{\tbltimeda};
    \addplot table[x index=0,y index=2]{\tbltimeda};
    \addplot table[x index=0,y index=3]{\tbltimeda};
    \draw [gray, very thin]
    ({axis cs:125,0}|-{rel axis cs:0,0}) -- 
    ({axis cs:125,0}|-{rel axis cs:0,1});
  \end{axis}
\end{tikzpicture} 
&
\pgfplotstableread{result/eik3d/strategy.dat}{\tbltimedb}
\begin{tikzpicture}
  \begin{axis}[
      ybar, ymin=0, bar width=6pt,
      width=0.33\textwidth,
      height=36mm,
      xtick=data,
      xtick align=inside,
      xticklabels from table={\tbltimedb}{Schedule},
      enlarge x limits={0.5}
    ]
    \addplot table [x expr=\coordindex+1,y=Vertex  ] {\tbltimedb};
    \addplot table [x expr=\coordindex+1,y=VertexCA] {\tbltimedb};
    \addplot table [x expr=\coordindex+1,y=BlockS  ] {\tbltimedb};
    \addplot table [x expr=\coordindex+1,y=BlockCvg] {\tbltimedb};
  \end{axis}
\end{tikzpicture} 
&
\pgfplotstableread{result/eik3d/maxinner.dat}{\tbltimedc}
\begin{tikzpicture}
  \begin{axis}[
      ymin={0},
      xtick=data,
      width={0.3\textwidth},
      height={36mm}]
    \addplot table[x index=0,y=Static   ]{\tbltimedc};
%    \addplot table[x index=0,y=StaticCvg]{\tbltimedc};
    \addplot table[x index=0,y=Eager    ]{\tbltimedc};
%    \addplot table[x index=0,y=EagerCvg ]{\tbltimedc};
    \addplot table[x index=0,y=Prior    ]{\tbltimedc};
%    \addplot table[x index=0,y=PriorCvg ]{\tbltimedc};
  \end{axis}
\end{tikzpicture}
\\[1mm] \hline \\[1mm]
%
% =====================================================================
% PPR(UK)
%
\begin{sideways} 
  \quad\quad 
  \parbox{18mm}{
    PPR(UK) \\
    {\scriptsize (Default BS=400)}
  }
\end{sideways} 
&
\pgfplotstableread{result/ukpr/blocksize.dat}{\tbltimeea}
\begin{tikzpicture}
  \begin{axis}[
      ylabel={Run time (s)}, ylabel near ticks,
      ymin={0},
      width={0.3\textwidth},
      height={36mm},
    ]
    \addplot table[x index=0,y index=1]{\tbltimeea};
    \addplot table[x index=0,y index=2]{\tbltimeea};
    \addplot table[x index=0,y index=3]{\tbltimeea};
    \draw [gray, very thin]
    ({axis cs:400,0}|-{rel axis cs:0,0}) -- 
    ({axis cs:400,0}|-{rel axis cs:0,1});
  \end{axis}
\end{tikzpicture} 
&
\pgfplotstableread{result/ukpr/strategy.dat}{\tbltimeeb}
\begin{tikzpicture}
  \begin{axis}[
      %cycle list shift=1,
      ybar, ymin=0, bar width=6pt,
      width=0.33\textwidth,
      height=36mm,
      xtick=data,
      xtick align=inside,
      xticklabels from table={\tbltimeeb}{Schedule},
      enlarge x limits={0.3}
    ]
    \addplot table [x expr=\coordindex+1,y=Vertex  ] {\tbltimeeb};
    \addplot table [x expr=\coordindex+1,y=VertexCA] {\tbltimeeb};
    \addplot table [x expr=\coordindex+1,y=BlockS  ] {\tbltimeeb};
    \addplot table [x expr=\coordindex+1,y=BlockCvg] {\tbltimeeb};
  \end{axis}
\end{tikzpicture} 
&
\pgfplotstableread{result/ukpr/maxinner.dat}{\tbltimecc}
\begin{tikzpicture}
  \begin{axis}[
      ymin={0},
      xtick=data,
      width={0.3\textwidth},
      height={36mm}
    ]
    \addplot table[x index=0,y=Static   ]{\tbltimecc};
%    \addplot table[x index=0,y=StaticCvg]{\tbltimecc};
    \addplot table[x index=0,y=Eager    ]{\tbltimecc};
%    \addplot table[x index=0,y=EagerCvg ]{\tbltimecc};
    \addplot table[x index=0,y=Prior    ]{\tbltimecc};
%    \addplot table[x index=0,y=PriorCvg ]{\tbltimecc};
  \end{axis}
\end{tikzpicture}
\\[1mm] \hline \\[1mm]
%
% =====================================================================
% Legends
%
& \ref{namedeffect} & \ref{namedb} & \ref{namedc}
\end{tabular}

\caption{Timing details for five application scenarios.}
\label{fig:timingGrid}
\end{figure*}

\begin{figure*}[ht]
  % -- DBLP
\pgfplotstableread{result/personalpr/dynamic.dat}{\tblcompa}
\begin{tikzpicture}
  \begin{axis}[
      title={PPR(Google)},
      ybar, ymin=0, bar width=6pt,
      width=0.3\textwidth,
      height=36mm,
      xtick=data,
      xticklabels from table={\tblcompa}{Outer},
      enlarge x limits={0.5},
      ylabel={Run time (s)}, ylabel near ticks,
      legend entries={StaticInner, DynamicInner},
      legend to name=nameddyinner]
    \addplot table [x expr=\coordindex+1,y=StaticInner] {\tblcompa};
    \addplot table [x expr=\coordindex+1,y=DynamicInner] {\tblcompa};
  \end{axis}
\end{tikzpicture}%
\hspace{3mm}
%
\pgfplotstableread{result/sssp/dynamic.dat}{\tblcompb}
\begin{tikzpicture}
  \begin{axis}[
      title={SSSP},
      ybar, ymin=0, bar width=6pt,
      width=0.3\textwidth,
      height=36mm,
      xtick=data,
      xticklabels from table={\tblcompb}{Outer},
      enlarge x limits={0.3}]
    \addplot table [x expr=\coordindex+1,y=StaticInner] {\tblcompb};
    \addplot table [x expr=\coordindex+1,y=DynamicInner] {\tblcompb};
  \end{axis}
\end{tikzpicture}%
\hspace{3mm}
%
\pgfplotstableread{result/eik3d/dynamic.dat}{\tblcompc}
\begin{tikzpicture}
  \begin{axis}[
      title={EtchSim},
      ybar, ymin=0, bar width=6pt,
      width=0.3\textwidth,
      height=36mm,
      xtick=data,
      xticklabels from table={\tblcompc}{Outer},
      enlarge x limits={0.3}]
    \addplot table [x expr=\coordindex+1,y=StaticInner] {\tblcompc};
    \addplot table [x expr=\coordindex+1,y=DynamicInner] {\tblcompc};
  \end{axis}
\end{tikzpicture}
%
\hspace{5mm}
\raisebox{1cm}{\ref{nameddyinner}}


  \caption{Effect of dynamic inner scheduling with different block level scheduling policies.}
  \label{fig:dynamicinner}
\end{figure*}


\end{document}







